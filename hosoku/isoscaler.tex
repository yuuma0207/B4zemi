%ctrl+alt+m:open Math Preview
\documentclass[a4paper,11pt,uplatex]{jsarticle}%titlepage
%:/usr/local/texlive/texmf-local/tex/latex/report/report.sty
\usepackage{myreport}
\title{2粒子系のアイソスカラーな演算子がドット積と単位演算子のみで書けることの証明}
\author{20B01392 松本侑真}
\date{\today}
\begin{document}
\maketitle
\begin{abstract}

\end{abstract}
\tableofcontents
\newpage
\section{証明したいこと}
\vskip\baselineskip
\begin{tcolorbox}[
    colback = white,
    colframe = green!35!black,
    fonttitle = \bfseries]
  $\mathcal{H}_{\tau_1}$と$\mathcal{H}_{\tau_2}$を粒子1と粒子2のアイソスピンが張るHilbert空間とする。
  合成系のアイソスピン空間$\mathcal{H}_{\tau_1}\otimes\mathcal{H}_{\tau_2}$上のエルミート演算子で、アイソスカラーなものを$\hat{A}$とする。
  このとき、$\hat{A}$は単位演算子$\hat{I}$と、アイソスピン演算子のドット積$\tau_1\cdot\tau_2$のみの関数である。
\end{tcolorbox}




\end{document}