\documentclass[a4paper,11pt,uplatex]{jsarticle}
\usepackage{report}
\renewcommand{\thesection}{3-\arabic{section}}
\title{Chapter3}
\author{20B01392 松本侑真}
\date{\today}
\begin{document}
\maketitle
\begin{abstract}

\end{abstract}
\tableofcontents
\newpage

\setcounter{section}{5}
\section{Yukawa Theory of Nuclear Interaction}
湯川が1934年に提唱した中間子の交換というアイデアは、以前の章の対称的な議論を通して得られたものを超えて、
核子間相互作用を調べるのには良いスタート地点である。
湯川の描像では、2つの核子間の相互作用は中間子の交換によって仲介される。
クォークの構造が根底にある、ハドロンの量子的な繋がりを率直に描いてはいないが、
その理論は中間子と核子間相互作用の強さのように、核子の相互作用を他のいろいろなハドロン原子の過程に結びつけることができる。
より実証的なレベルでは、湯川のアイデアは核子のポテンシャルの距離依存性の合理的なモデルを提供してくれる。
そのような表現は、例えば、半経験的なアプローチ方法として使えるかもしれない。
本章における焦点は、主に中間子交換のアイデアの起源そのものである。
応用については最終章に残すことにして、核子-核子散乱から提供される実験的な情報を最初に確認する。

ボソンの交換に基づくポテンシャルの適切な導出は、今扱っている範疇を超えている、相対論的な場の量子論を扱う必要がある。
しかしながら、そのエッセンスは古典的な電気力学とのアナロジーを考えることで得られる。
湧き出しのない場所の静電ポテンシャル$\phi(\bm{r})$は、ラプラス方程式の解となっている:
\begin{equation}
  \laplacian\phi(\bm{r})=0\;。
  \label{eq:laplas}
\end{equation}
電荷$q$の点電荷が原点に存在しているとき、方程式は
\begin{equation}
  \laplacian\phi(\bm{r})=-\qty[\frac{1}{4\pi\varepsilon_0}]4\pi q\delta(\bm{r})
  \label{eq:poisson}
\end{equation}
となる。この解として、良く知られているクーロンポテンシャルが得られる:
\begin{equation}
  \phi(\bm{r})=\qty[\frac{1}{4\pi\varepsilon_0}]\frac{q}{r}\;。
  \label{eq:3-59}
\end{equation}
電磁場が量子化されるとき、場の量子として光子が出現し、電荷は場の源となる。(どういう意味ですか?)
\vskip\baselineskip
\begin{tcolorbox}[
    colback = white,
    colframe = green!35!black,
    fonttitle = \bfseries]
  \begin{theorem}[Poisson方程式の解]
    一般に、Poisson方程式は
    \begin{equation}
      \laplacian u(\bm{r})=-f(\bm{r})
    \end{equation}
    と表される。このとき、解は
    \begin{equation}
      u(\bm{r})=\frac{1}{4\pi}\iiint d^3\bm{r}'\,\frac{f(\bm{r}')}{\abs{\bm{r}-\bm{r}'}}
    \end{equation}
    である。
  \end{theorem}
\end{tcolorbox}
核力は電磁気的な力とは異なる点がいくつかある。
恐らく最も重要な点は、それが短距離力ということであり、次章でそれを支持する証拠を確認する。
今は、短距離の核ポテンシャルに対する場の量子論における類似物である
式\eqref{eq:poisson}と似た式の問題を主に考えることにする。
考える式は、相対論的極限でも成立するために、ローレンツ共変でなければならない。
シュレディンガー方程式は非相対論的極限でのみ適用されるため、このルールを満たしていない。
核子を交換している場の量子はボソンでなければならない。なぜなら、ボソンのみが単体で生成消滅することが可能だからである。
フェルミオンは一方で、その反粒子とともに生成消滅する。ディラック方程式はスピン$1/2$の粒子(言い換えれば、フェルミオン)を記述する方程式であるため不適当である。
第一候補としては、クライン・ゴルドン方程式が残されている。

相対論的なエネルギー運動量関係は、
\begin{equation}
  E^2=p^2c^2+m^2c^4
\end{equation}
で与えられる。
エネルギー$E$を演算子$i\hbar(\pdv*{t})$に、運動量$\bm{p}$を演算子$-i\hbar\grad$に置き換えることで、
非相対論的な量子力学を得たときと同様の方法で量子化することができる:
\begin{equation}
  -\hbar^2\pdv[2]{t}\phi(\bm{r}) = \qty(-\hbar^2c^2\laplacian + m^2c^4)\phi(\bm{r})\;。
\end{equation}
ここで、$m$は場の量子の質量とする。両辺に$(\hbar c)^2$をかけて整理すると、良く知られたクライン・ゴルドン方程式を得る:
\begin{equation}
  \qty(\laplacian - \frac{1}{c^2}\pdv[2]{t})\phi(\bm{r}) = \frac{m^2c^2}{\hbar^2}\phi(\bm{r})\;。
  \label{eq:3-61}
\end{equation}
これは、場の量子のソース項はまだ含まれていないため、式\eqref{eq:laplas}のアナロジーにすぎない。
この点は、場の量子の質量$m$をゼロにして、時間依存性を無視することでさらに実証されるかもしれない。
その結果は、式\eqref{eq:laplas}と同じである。

湧き出しを含めるためには、式\eqref{eq:3-61}にソース項を追加して、ポアソン方程式\eqref{eq:poisson}に相当するものを求める必要がある。
簡単化のため、静止限界(static limit)のみを考えて、時間微分を含む項を無視する。
原点に強さが$g$の点源があるとき、
\begin{equation}
  \laplacian\phi(\bm{r}) = \frac{m^2c^2}{\hbar^2}\phi(\bm{r}) - g\delta(\bm{r})
  \label{eq:3-69}
\end{equation}
を得る。この式の解
\begin{equation}
  \phi(\bm{r})=\frac{g}{4\pi r}e^{-mcr/\hbar}
\end{equation}
は、湯川ポテンシャルの形として良く知られており、$m=0,\,g=\qty[(4\pi\varepsilon_0)^{-1}]4\pi q$とすることで、
式\eqref{eq:3-59}に帰着できる。
一方で、もし場の量子に有限の質量があると、ポテンシャルの強さが$r_0=\hbar/mc$で$1/e$倍に落ちることがわかる。
$r_0$は、質量$m$のボソンが仲介する力の範囲の尺度として捉えることができる。パイ中間子($m\sim \SI{140}{MeV/c^2}$)では、
$r_0$の値はおよそ\SI{1.4}{fm}である。後で見るように、1つのパイ中間子の交換は、核力ポテンシャルの長距離部分をよく表現している。

\vskip\baselineskip
\begin{tcolorbox}[
    colback = white,
    colframe = green!35!black,
    fonttitle = \bfseries]
  \begin{theorem}[湯川ポテンシャルの導出]
    点源がある場合のクライン・ゴルドン方程式
    \begin{equation}
      \laplacian\phi(\bm{r}) = \frac{m^2c^2}{\hbar^2}\phi(\bm{r}) - g\delta(\bm{r})
    \end{equation}
    のFourier変換を考える。
    以下では、関数$f(\bm{r})$のFourier変換を
    \begin{equation}
      \mathcal{F}[f]\coloneqq \frac{1}{(2\pi)^3}\int_{-\infty}^{\infty} d^3\bm{r} f(\bm{r})e^{-i\bm{k}\cdot\bm{r}}
    \end{equation}
    と定義する。
    $\mathcal{F}[\laplacian f] = -\abs{\bm{k}}^2\mathcal{F}[f]$
    となることは部分積分から容易に示すことができるため、式\eqref{eq:3-69}の両辺をFourier変換すると
    \begin{equation}
      \mathcal{F}[\phi] = \frac{g}{(2\pi)^3}\frac{1}{k^2+m^2c^2/\hbar^2}
    \end{equation}
    となる。したがって、逆Fourier変換は以下のようになる:
    \begin{equation}
      \phi(\bm{r}) = \frac{g}{(2\pi)^3}\int_0^\infty dk\frac{k^2}{k^2+m^2c^2/\hbar^2}\int_{-1}^{1}d(\cos\theta) e^{ikr\cos\theta}\int_{0}^{2\pi}d\varphi \,。
    \end{equation}
    \begin{equation}
      \int_{-1}^{1}d(\cos\theta) e^{ikr\cos\theta} = \frac{1}{ikr}\qty[e^{ikr} - e^{-ikr}]
    \end{equation}
    となり、第二項目は第一項目で$k\to-k$とした場合に対応しているため、
    \begin{equation}
      \phi(\bm{r}) = \frac{g}{(2\pi)^2ir}\int_{-\infty}^{\infty}dk\frac{ke^{ikr}}{k^2+m^2c^2/\hbar^2}
    \end{equation}
    となる。$r>0$であるため、複素平面の上半分を通る半径$R$の半円経路を考えると、$R\to\infty$で半円上での積分は$0$に収束する。したがって、
    \begin{equation}
      \phi(\bm{r}) = \left.\frac{g}{2\pi r}\frac{ke^{ikr}}{(k^2+m^2c^2/\hbar^2)'}\right|_{k=imc/\hbar} = \frac{g}{4\pi r}e^{-mcr/\hbar}
    \end{equation}
    を得る。
  \end{theorem}
\end{tcolorbox}

\newpage
\section{Nucleon-Nucleon Scattering Phase Shifts}
核子-核子相互作用によるポテンシャルの表式
\begin{align}
  V(r;\,\bm{\sigma}_1,\,\bm{\sigma}_2,\,\bm{\tau}_1,\,\bm{\tau}_2) \;=\; &
  V_0(r)+V_\sigma(r)\bm{\sigma}_1\cdot\bm{\sigma}_2 + V_\tau(r)\bm{\tau}_1\cdot\bm{\tau}_2
  + V_{\sigma\tau}(r)\qty(\bm{\sigma}_1\cdot\bm{\sigma}_2)\qty(\bm{\tau}_1\cdot\bm{\tau}_2) \notag                                                    \\
                                                                         & +V_{LS}(r)\bm{L}\cdot\bm{S} +
  V_{LS\tau}(r)\qty(\bm{L}\cdot\bm{S})\qty(\bm{\tau}_1\cdot\bm{\tau}_2) \notag                                                                        \\
                                                                         & +V_{T}(r)S_{12}+V_{T\tau}(r)S_{12}\qty(\bm{\tau}_1\cdot\bm{\tau}_2) \notag \\
                                                                         & +V_Q(r)Q_{12}+V_{Q\tau}(r)Q_{12}\bm{\tau}_1\cdot\bm{\tau}_2 \notag         \\
                                                                         & + V_{PP}(r)\qty(\bm{\sigma}_1\cdot\bm{p})\qty(\bm{\sigma}_2\cdot\bm{p})
  +V_{PP\tau}(r)\qty(\bm{\sigma}_1\cdot\bm{p})\qty(\bm{\sigma}_2\cdot\bm{p})\qty(\bm{\tau}_1\cdot\bm{\tau}_2)
\end{align}
は、重陽子の方法と、二核子系の対称性を用いて得られた。さらなる飛躍のためには追加の実験的な情報が必要であり、ある核子が異なるエネルギーを持つ別の核子から散乱されることでもたらされる。

\subsection*{Nucleon-nucleon scattering}
原理的に、2つの核子を含む4種類の散乱測定が可能である。入射陽子が陽子によって散乱されること($pp$-scattering)は、実験的観点から、4つのうちで最も単純である。
なぜなら、陽子を加速や、水素(陽子)を含む標的の構築が比較的簡単であるためである。中性子散乱では、入射ビームは大まかに2つの源がある。
低エネルギーでは、原子炉中の中性子が使われる。高エネルギーでは、例えば、\ce{^{7}Li}を標的とする$(p,\,n)$反応のように、陽子ビームから作られる中性子を使うことができる。
しかしながら、これらの方法で得られる中性子ビームの強度やエネルギー分解能は、陽子ビームのそれよりもはるかに制限されている。
その結果、中性子散乱は、一般的に、陽子散乱よりも難しい実験である。
陽子標的からの中性子散乱($np$-scattering)と、それに準ずる$pn$散乱は、$T=0$チャンネルでも反応が生じるという点で重要である。
対称的に、$pp-$と$nn-$散乱は2核子系の$T=1$状態の情報しか提供しない。(なぜ?)





\end{document}