\RequirePackage{plautopatch}
\documentclass[10pt,aspectratio=169,xcolor=dvipsnames,table,dvipdfmx]{beamer}
\usepackage{mybeamer}
%タイトル
\title{Feynman-Hibbs「経路積分と量子力学」}
\subtitle{第一章 量子力学の基本概念}
\author{\textbf{松本侑真}}
\date{\today}

\begin{document}
\maketitle

\begin{frame}{量子力学における確率}
  20世紀の初め頃から、実験物理学の分やで古典力学の妥当性を否定するような奇妙な現象が続々と蓄積されていた。
  これらの新しい現象を説明できるような新しい理論の枠組みでは、電子が波動・粒子の両方の性質を持つように見える。
  このような事態は1926年、1927年に{\color{red}量子力学}と呼ばれる理論によって完全に解決された。
  \begin{block}{量子力学による予言}
      量子力学によると、
    \begin{itemize}
      \item 精密な結果を予測することが根本的に不可能な実験が存在するため、
      いろいろな確率の確率を計算することで満足しなければならない。
      \item 自然の法則の中にある確率は、Laplaceの古典確率論における確率組み合わせの放送と異なるものである。
      \item 量子力学の法則は、実験の対象の大きさが増大すればするほど、Laplaceの法則に近くなる。
    \end{itemize}
      ということがわかってきた。
  \end{block}
\end{frame}

\begin{frame}{1つの概念上の実験}
  二重スリット実験は、量子力学の効果が現れる最も単純な実験の1つである。
  電子が粒子のように振舞うということから、
  \begin{itemize}
    \item 線源$S$からスクリーン上のとある位置$x$に進む電子はそれぞれ穴1、穴2どちらか一方を通る。
    \item その帰結として、$x$に到達する確率は、穴1を通って到達する確率$P_1$と、穴2を通って到達する確率$P_2$の和に等しい
    と予想される。
  \end{itemize}
  しかし、実際の実験では、両方の穴が開いているときに電子が$x$に到達する確率$P(x)$に対して、
  \begin{equation}
    P(x) \neq P_1 + P_2
  \end{equation}
  となっている。そのため、先ほどの予想は誤りであることになる。
\end{frame}

\begin{frame}
  \frametitle{確率振幅}
  \begin{itemize}
    \item   両方の穴が開いているときに$x$に到達する確率は、片方の穴のみが開いているときの確率の和ではない。
    実際に、$P(x)$は線源$S$を出発した波が2つの穴を通ってスクリーンに到達したときに生じる干渉パターンの強度分布として知られている。
    \item 波動を表現するのに最も簡単な方法は複素数を用いることであり、$P(x)$は、$x$に到達することを表すある複素数$\phi(x)$の絶対値の2乗として表現される。
  \end{itemize}
  \begin{block}{2重スリット実験に関する法則}
    複素数$\phi_1,\,\phi_2$に対して、
    \begin{equation}
      P = \abs{\phi}^2 = \abs{\phi_1 + \phi_2}^2 ,\quad P_1 = \abs{\phi_1}^2 ,\quad P_2 = \abs{\phi_2}^2
    \end{equation}
    が成り立つ。
  \end{block}
  $\phi_1$は、ある波動方程式に対して、線源$S$から穴1に伝播し、そして穴1から$x$へと伝播する波を表現する解として求められる。
  つまり、$\phi_1$は電子(光の場合は光子)の波動性を表している。\\
  \textbf{まとめ: }\color{red}われわれは$x$に在る検出装置に到達する波の強度を計算し、その強度を粒子の到達確率として解釈する。
\end{frame}

\begin{frame}
  \frametitle{2重スリット実験の考察}
  \begin{block}{論理的困難}
    波動性と粒子性を両方用いることは何ら矛盾を生じない。ただし、実験の状況についてどのような言明が許されるかには細心の注意が必要である。
\begin{itemize}
  \item 2重スリット実験に関する法則によると、一般に$P=P_1+P_2$は成り立たない。
  \item このことから、2つの穴が開いているとき、粒子がどちらか一方の穴を通ることは正しくないという結論にせざるを得ない。
  \item 仮に、粒子がどちらか一方の穴を通るということが正しいとすると、穴1と穴2を通るのは排反事象であることから、$P=P_1+P_2$が成立してしまう。
\end{itemize}
  \end{block}
古典的描像に基づく論理的困難を乗り越えるために、様々な実験の工夫を行うことができる。
例えば、光子を用いる場合では、干渉性を持つ2つの経路を何センチも大きく話すことによって、2つの光の経路をほとんど確かに独立にすることができる。
しかし、実際の状況が想像以上に深遠であることは、次の実験によって明らかとなる。
\end{frame}

\begin{frame}
  \frametitle{電子の経路の観測}
  \begin{itemize}
    \item 電子がどちらか一方の穴を通るのではないということを検証することは、実験の工夫によって可能である。
    \item 電子は光を散乱するため、穴の後方に光源を置いて光の散乱位置を観測すれば良い。
  \end{itemize}
  \begin{block}{観測の効果}
    電子がどちらの穴を通るかを観測し、同時に電子が$x$に到達する確率を測定してみる。
    こうすれば、$x$に到達する電子が穴1、穴2のどちらを通ってきたのかを実験的に確かめることが可能になる。その結果、
    \begin{itemize}
      \item 穴1を通る電子は、$P_1$で与えられることが確かめられる。
      なぜなら、$x$に到達する電子のうちで、穴1を通ったと思われる(光の散乱から確かめられる)電子のみを選べば、
      確かに$P_1$の分布になっていることがわかるからである。この結果は、穴2が閉じているかどうかによらず得られるものであり、穴2の存在は穴1の近くの運動に影響を与えない。
      \item 穴2を通る電子についても同様である。
      \item それらの電子を一緒にすれば、$P=P_1+P_2$が得られるはずであり、実際に実験すればそうなる。
      この場合、$P = \abs{\phi_1+\phi_2}^2$とはならない。
    \end{itemize}
  \end{block}
\end{frame}

\begin{frame}
  \frametitle{観測の結果の影響について}
  \begin{block}{光による撹乱}
    \begin{itemize}
      \item 光はエネルギー$h\nu$、運動量$h/\lambda$の光子として入射してくる。光を弱くするということは光子の数を少なくするということであり、散乱して観測できる電子の個数が減るというだけである。
      \item 電子を見出した時には、完全に1個の光子が散乱されたことになり、$h/\lambda$程度の有限な運動量が電子に与えられることになる。
      \item 観測された電子は$P_1+P_2$、観測されなかった電子は$\abs{\phi_1+\phi_2}^2$としてスクリーン上に分布する。
    \end{itemize}
    この場合、全体の分布はそれぞれの分布の加重平均($p$は観測される電子の個数に比例)
    \begin{equation}
      P = (1-p)\abs{\phi_1+\phi_2}^2 + p(P_1+P_2)
    \end{equation}
    となる。また、$\lambda$を長くすれば撹乱は小さくなるが、波長$\lambda$の光源の位置を$\lambda$の程度よりも精密に確定することができない。
    したがって、電子が通った穴の位置を決定することもできない。
  \end{block}
\end{frame}

\begin{frame}
  \frametitle{不確定性原理}
  Heisenbergは、「新しい力学が無矛盾であるためには実験を行う際の撹乱を小さくするのに限界がある」と述べ、観測による撹乱の影響に初めて気づいた。
  \begin{block}{Heisenbergによる不確定性原理の表現}
    \begin{itemize}
      \item 古典物理学では、粒子はある特定の軌道に沿って運動するものと記述され、任意の時刻においてたとえば位置と運動量が正確に与えられる。
      \item しかし、量子力学に従う系(現実の系)では、古典物理学的見地から見ると、位置もしくは運動量が特定できないで不確定な系として見える。
      \item 別の測定手段によって位置や運動量片方の不確定性を減らすことはできるが、両者を同時
      \footnote{ここでいう「同時」とは、「同時刻」ではなく、「2つの物理量の同時確率分布に対してそれぞれの周辺確率分布が正しく定まるかどうか」ということだと思います。}
      に正確に知ることはできないのである。
      \item どのような実験においても、運動量と位置の不確定性の積は$\hbar$の程度よりも小さくなることはない。
    \end{itemize}
  \end{block}
\end{frame}

\begin{frame}
  \frametitle{不確定性原理についての背景}
1927年にHeisenbergの思考実験によって初めて不確定性原理が導かれた。そのときは、
\begin{equation}
  \varDelta x \sim \lambda,\, \varDelta p \sim \frac{h}{\lambda}\Rightarrow \varDelta x \varDelta p \sim h
\end{equation}
という大雑把な見積もりである。
しばらくして、ケナードの不等式
\begin{equation}
  \sigma(Q)\sigma(P)\geq \frac{\hbar}{2}\qq{($\sigma(Q)$は$Q$の標準偏差)}
\end{equation}
が数学的に導かれた。そのため、Heisenbergはこの不等式に影響を受けて
\begin{equation}
  \varDelta x \varDelta p \geq \frac{\hbar}{2}\qq{(間違い)}
\end{equation}
と1930年に論文を書いた。これを現代的(より正確な表現を用いて)書くと、
\begin{equation}
  \epsilon(Q)\eta(P)\geq \frac{\hbar}{2}\qq{(間違い)}
\end{equation}
となるが、この式は厳密に間違っている。ここで、$\epsilon(X)$は物理量$X$の誤差、$\eta(X)$を物理量$X$の擾乱である。
\end{frame}

\begin{frame}
  \frametitle{ケナードの不等式の意味}
  \begin{equation}
    \sigma(Q)\sigma(P)\geq \frac{\hbar}{2}\qq{(ケナードの不等式)}
  \end{equation}
  \begin{itemize}
    \item  同一の量子状態を大量(例えば1000個)用意する。このうちの500個では$Q$の測定、残りの500個では$P$の測定を行う。
    \item 誤差がゼロの理想的な測定ができたとする。
    \item  しかし、どんな量子状態を持ってきても、必ず$Q,\,P$どちらかの物理量の標準偏差(統計的なばらつき)が存在するという主張がケナードの不等式。(厳密に正しい)
  \end{itemize}
  この測定は、別々の(同じ)量子状態を用意してそれぞれに対して測定を行っている。
  同じ系で$P$を測定した後に$Q$を測定するといった同時測定をしているわけではない。
\end{frame}

\begin{frame}
  \frametitle{小澤の不等式(2003年)}
  \begin{equation}
    \epsilon(Q)\eta(P)+\epsilon(Q)\sigma(P)+\sigma(Q)\eta(P)\geq \frac{\hbar}{2}
  \end{equation}
  中性子のスピンを使った実験によって、小澤の不等式が実験的に証明された。
  この実験は、Heisenbergの不確定性原理を破ったとして話題になった。実際に、
  \begin{equation}
    \epsilon(Q)\eta(P)\geq \frac{\hbar}{2}-\qty(\epsilon(Q)\sigma(P)+\sigma(Q)\eta(P))
  \end{equation}
  と式変形ができるが、真に$\epsilon(Q)\eta(P) < {\hbar}/{2}$となった実験であるからである。
  
  小澤の不等式では、等号成立条件がタイトなものであるかどうかは別問題である。
  例えば、自明な場合でしか等号が成立しないのであれば、その場合を除いて、より範囲を狭めた不等式が構成できる。
  そこで、小澤の不等式の等号条件の必要十分条件を明らかにするか、小澤が定義した誤差や擾乱の関数でなるべくタイトな不等式を見つけることが課題となる。
  
  ブランシアードは、小澤の不等式に替わる最もタイトな不等式を見出し、その等号成立条件も明らかにした。
  実際に、ブランシアードの不等式から小澤の不等式を導くこともできる。
  
  また、小澤が定義した誤差や擾乱の定義を、より物理的に意味のある(?)定義に直して、渡辺・沙川・上田の不等式が構築された。
\end{frame}


\begin{frame}{不確定性原理が述べていること(測定誤差と測定誤差の反作用)}
    \begin{itemize}
      \item スクリーンCの穴1と穴2を通って$x$に到達する電子では、軌道が曲がることによる運動量の変化分が異なる。
      それぞれの穴を通るときの運動量の変化分の差を$\delta p$とする。
      \item $C$が上下方向に自由に動けるものとして、その運動量変化の大きさを$\delta p$以上の精度で測定すれば、電子がどちらの穴を通ったかがわかる。
      \item 実際にその精度で測定できる装置を用いれば、干渉は$P=P_1+P_2$となる。では、なぜ$P=\abs{\phi_1+\phi_2}^2$とならないのか?
      \begin{itemize}
      \item $C$が上下に$\delta x$だけ動くと、電子の干渉面への到達位置は上下に$\delta x$だけ変化する。
      \item その結果、干渉面の位置$x$の上下$\delta x$の間の平均値として干渉パターンが観測されるため、到達位置の較正をしないと$C$を固定した場合と同じ分布が得られない。
      \item すなわち、スクリーン$C$の位置の変化$\delta x$は、干渉縞の極大点の幅$d$の半分以上の精度$(\delta x <d/2)$で決定してスクリーン$C$のゼロ点を較正する必要がある。
      \item 不確定性原理が正しいとすると、スクリーンの運動量変化を$\delta p$以上の精度で測定できる場合は、$\delta x$を$d/2$以上の精度で測定できないことを意味している。
      \end{itemize}
    \end{itemize}
  以上で不確定性原理が証明されたわけではないが、実験的に不確定性原理が破れていることが示されておらず、
  また、現実世界を高い精度で予言している量子力学の法則が首尾一貫したものであるためには、不確定性が要求されるように思われている。
\end{frame}

\begin{frame}
  \frametitle{スクリーンの位置による干渉縞の消失}
\begin{center}
  \adjincludegraphics[height=0.5\linewidth]{pic/hukakutei.jpg}
\end{center}
\end{frame}

\begin{frame}
  \frametitle{確率概念のまとめ}
  線源$S$から検出器の位置$x$へ進む電子は、スクリーンC上の穴1を通って$x$に到達する場合の振幅と、
  穴2を通って到達する場合との2つの振幅を持つ。さらに、それぞれの選択肢の振幅を加えることにより、全部の事象に対応させることができる。
  たとえば、$x$に到達することの全振幅は
  \begin{equation}
    \phi(x) = \phi_1(x) + \phi_2(x)
  \end{equation}
  であり、その事象の起こる確率は
  \begin{equation}
    P(x) = \abs{\phi(x)}^2 = \abs{\phi_1(x)+\phi_2(x)}^2
  \end{equation}
  である。その事象が完了する前に、粒子の状態を観測してしまうと、その全振幅の構成を撹乱することになる。
  すなわち、粒子の系が他の状態にある可能性を排除したことになり、排除された状態に付随した確率振幅は全振幅の計算には加えられない。
\end{frame}
\begin{frame}
  \frametitle{スクリーンにさらに穴を開け、スクリーンの数も増やす}
  \begin{itemize}
    \item 先ほどの考え方を拡張して、地点AB間に置くスクリーンの数を増やし、スクリーン自体の穴もさらに開けると、それぞれの穴を通る振幅の和を考える必要がある。
    \item 例えば、位置$y_E$にあるスクリーン$E$の高さ$x_E$の穴から出て、位置$y_D$にあるスクリーン$D$の高さ$x_D$の穴に入る電子の振幅を考える必要がある。
    \item すなわち、$y_E$から$y_D$への経路の選択肢の和は、位置$y_D,\,y_E$にある2つのスクリーンを通過する高さを記述する連続パラメータ$x_D,\,x_E$についての積分によって与えられる。
    \item スクリーンの数も無限個に増やすことで、地点ABを結ぶ経路の和は、$y_A\leq y\leq y_B$と$x_A\leq x\leq x_B$による二重積分で表現される。
  \end{itemize}
  また、経路$x(y)$上の区間の各点を通る時刻も考えることができる。
  すなわち2次元空間であれば、2つの関数$x(t),\,y(t)$が与えられれば1つの経路が決定される。
  このようにいて、特定の経路$x(t),\,y(t)$をとる振幅という概念が得られる。
\end{frame}

\frame{\centering{\Huge{Thank you for your attention!}}}

\begin{frame}
  \frametitle{Heisenberg-Robertson型の不確定性関係}
$A,\,B$を任意の物理量、$\ket{\psi}$を任意の状態とする。このとき、
\begin{equation}
  \varDelta_{\psi}[A]\varDelta_{\psi}[B]\geq \frac{1}{2}\abs{\bra{\psi}[A,B]\ket{\psi}}
\end{equation}
が成り立つ。ここで、$\varDelta_{\psi}[X] = \sqrt{V_{\psi}[X]}$であり、
$\varDelta_{\psi}[X] = \ev{X^2}{\psi} - \ev{X}{\psi}^2$
である。また、$[A,B] = AB-BA$である。特に、$[Q,P]=i\hbar$を代入すると、ケナードの不等式が導ける。

\begin{itemize}
  \item 不確定性関係は、非可換な物理量どうしの同時測定が存在しないことを示している。
  \item 同時測定とは、2つの物理量の同時確率分布に対してそれぞれの周辺分布がBornの確率規則を満たすこと。つまり、
  \begin{equation}
    \ev{P_a}{\psi} = \sum_{b\in B}\Pr(A=a,\,B=b\mid\ket{\psi}),\, \ev{Q_b}{\psi} = \sum_{a\in A}\Pr(A=a,\,B=b\mid\ket{\psi})
  \end{equation}
  が成り立つこと。ただし、$P_a,\,Q_b$は$a,\,b$への射影演算子であり、
  $\ket{\psi}$の状態における$A$の測定の結果が$a$であり、かつ、$B$の測定の結果が$b$である確率を$\Pr(A=a,\,B=b\mid\ket{\psi})$とした。
\end{itemize}
\end{frame}

\begin{frame}
  \frametitle{同時測定について}
  \begin{block}{命題:可換な物理量の同時測定}
    可換な物理量$A,\,B$は同時測定できる。特に、$\ket{\psi}$の元で$A,\,B$を同時測定したときの同時確率分布は
  \begin{equation}
    \Pr(A=a,\,B=b\mid\ket{\psi}) = \ev{P_aQ_b}{\psi}
  \end{equation}
  で与えられる。(証明略:同時固有状態が取れることを用いる。)逆も成り立つが、その証明にはPOVMの概念が必要になる。
  \end{block}
  \begin{block}{合成系の公理}
    Hilbert空間$\mathcal{H}_1,\,\mathcal{H}_2$が付随する量子系$S_1,\,S_2$の合成系には、テンソル積Hilbert空間$\mathcal{H}_1\otimes\,\mathcal{H}_2$が付随する。
    系$S_1$の物理量$A$は合成系では$A\otimes I_2$で表される。$I_2$は$\mathcal{H}_2$上の単位演算子である。$S_2$についても同様である。
  \end{block}
  \begin{block}{局所性の要請}
    系$S_1,\,S_2$が空間的に離れているとき、$A,\,B$の同時測定を「$A,\,B$それぞれを局所的に測定する」ことで実現できる。
    この測定は測定$A,\,B$の時間順序に依らない。
  \end{block}
  

\end{frame}

\begin{frame}
  \frametitle{一般的な測定を考えるための準備:密度演算子}
    密度演算子$\rho$の数学的な特徴づけは以下の2つ
    \begin{equation}
      \Tr(\rho)=1,\, \rho\geq 0\qq{(正値演算子)}
    \end{equation}
    である。物理的な特徴づけとしては、量子状態の準備の確率まで含めた一般的な状態の記述を考える。
    確率$p_i$で状態$\ket{\psi_i}$を準備するような状態$s$を$s=\{p_i\,;\ket{\psi_i}\}$と表すとする。
    このような操作を確率混合と言い、$s=\{1,\,\ket{\psi}\}$を純粋状態、そうでない場合を混合状態と呼ぶ。
    混合状態はHilbert空間の単位ベクトル($=$状態ベクトル)で表すことができないが、このような状態を記述するのが密度演算子である。
    状態$s$の元で物理量$A$を測定し、$a$を得る確率は
    \begin{equation}
      \Pr(A=a\mid s) = \sum_{i}p_i\Pr(A=a\mid\ket{\psi_i}) = \sum_{i}p_i\ev{P_a}{\psi_i}
    \end{equation}
    となる。ここで、密度演算子を
    \begin{equation}
      \rho = \sum_{i}p_i\ketbra{\psi_i}
    \end{equation}
    と定めると、$\Pr(A=a\mid s) = \Tr(P_a\rho)$となる。
    また、$s$を構成する確率混合は一意ではないが、$\rho$は$s$に対して一意的に定まる。
    さらに、$A$は任意であったため、$\rho$は系の状態を完全に記述していることがわかる。
    (全ての物理量で同じ確率分布を与える状態は同じ$\rho$)
\end{frame}

\begin{frame}
  \frametitle{一般的な測定が満たすべき性質:アフィン性}
    測定とは、状態に対して物理量の確率分布を与えるもの、すなわち、
    \begin{equation}
      f_m\colon \rho \mapsto \Pr(M=m\mid \rho)
    \end{equation}
    という関数$f_m$の集まりを測定と思ってよい。状態は確率混合(状態の凸結合)
    \begin{equation}
      \rho = p\rho_1 + (1-p)\rho_2
    \end{equation}
    に対して閉じているため、測定もこの構造を保つ必要があるだろう。すなわち、
    \begin{equation}
      \Pr(M=m\mid p\rho_1+(1-p)\rho_2) = p\Pr(M=m\mid \rho_1)+(1-p)\Pr(M=m\mid \rho_2)
    \end{equation}
    を満たさなければならない、と考えられる。
    このように、凸結合を保つような性質を一般にアフィン性といい、その写像をアフィン写像という。
    \begin{block}{測定$M$の必要条件}
      測定値の集合$\mathcal{M}$を有し、状態$\rho$の下でその測定を行うと、測定値$m\in\mathcal{M}$を得る確率が定まっており、
      アフィン性を満たす。
    \end{block}
\end{frame}

\begin{frame}
  \frametitle{POVM測定}
  \begin{alertblock}{量子系における測定}
    アフィン性を満たす任意の測定は、POVM測定で記述される。また、POVM測定はアフィン性を満たす。
    すなわち、POVM測定は量子系における測定の一般的な枠組みである。
  \end{alertblock}
  まず、POVM(Positive Operator Vaued Measure)とは、線形演算子の組$\{E_m\in \mathcal{L}(\mathcal{H})\}_{m\in \mathcal{M}}$であって、
  $E_m\geq 0$かつ$\sum_m E_m = I$を満たすものをいう。それぞれの$E_m$をPOVM要素という。
  
  \begin{block}{POVM測定(射影測定の一般化)}
    量子系での測定$M$がPOVM測定とは、状態$\rho$の下で、測定値$m\in \mathcal{M}$を得る確率が、とあるPOVM$\{E_m\}_{m\in\mathcal{M}}$
    を用いて
    \begin{equation}
      \Pr(M=m\mid \rho) = \Tr(E_m\rho)
    \end{equation}
    で与えられるものをいう。
  \end{block}
POVM測定の例として、POVMを射影演算子の組$\{P_m\}_{m\in \mathcal{M}}$としたものがある。これによるPOVM測定をPVM測定という。
\textbf{PVM測定はエルミート演算子の測定と等価である。}
\footnote{任意のPOVM測定は\textbf{間接測定}によって物理的に実現可能であることも示すことができる。
具体的には、着目系$S$の任意のPOVM$E$を固定したとき、任意の状態$\rho$に対して、
環境系のHilbert空間$\mathcal{H_A}$、環境系の(純粋)状態$\sigma$、全体系$\mathcal{H}_S\otimes \mathcal{H}_A$上のユニタリ演算子$U$、
間接測定に用いるメーター物理量$B=\sum_{m}mP_m\in \mathcal{L}(\mathcal{H}_A)$が存在する。}
\end{frame}

\begin{frame}
  \frametitle{POVMを用いた同時測定の必要十分条件}
  POVMを用いると、同時測定の定義を以下のように書き直すことができる。
  \begin{block}{POVMによる同時測定}
    POVM$E=\{E_m\}_{m\in\mathcal{M}},\,F=\{F_n\}_{n\in\mathcal{N}}$に対して、
    あるPOVM$K=\{K_{mn}\}_{m\in\mathcal{M},\,n\in\mathcal{N}}$が存在して、任意の状態$\rho$で
    \begin{equation}
      \Tr(\rho E_m) = \sum_{n\in\mathcal{N}}\Tr(\rho K_{mn}),\quad \Tr(\rho F_n) = \sum_{m\in\mathcal{M}}\Tr(\rho K_{mn})
    \end{equation}
    となることである。これは$\rho$が任意なので
    \begin{equation}
      E_m = \sum_{n}K_{mn},\quad F_n = \sum_{m}K_{mn}
    \end{equation}
    と等価である。
  \end{block}
  POVMを用いた同時測定によって、「物理量$A,\,B$が同時測定できるならば、物理量$A,\,B$は可換である」が示される。
  以上より同時測定の必要十分条件が示される:
  \begin{block}{定理:同時測定の必要十分条件}
    物理量$A,\,B$は可換であるとき、そのときに限り同時測定可能である。
  \end{block}
\end{frame}

\begin{frame}
  \frametitle{測定に関するあれこれのまとめ}
  \begin{itemize}
    \item Heisenbergの思考実験による位置と運動量の不確定性関係は、測定誤差と測定の反作用に由来するものである。(小澤の不等式など)
    \item 同じ系のコピーを大量に用意して誤差と反作用が存在しない理想測定ができたとしても、Robertson型の不確定性関係(ケナードの不等式)は成り立つ。
    \item 物理量の「同時測定」とは、「同時確率分布から求めたそれぞれの周辺分布がBornの確率規則に従う」ことであり、測定の「同時性」ではない。
    \item 同時測定ができるとき、そのときに限り、物理量は可換である。
    \item 量子論の枠組みの中で最も一般的な(測定後の状態の情報を気にしない)測定
    \footnote{ある物理量を測定した後に、続けて他の物理量を測定することを継続測定という。
    POVM測定では物理量を測定した後の状態の情報を得られないため、継続測定は行えない。
    測定後の状態まで情報を得るためには、CP-インストルメント(測定演算子)による測定を行う必要がある。
    Von Neumannの射影仮説は、物理量の測定後の状態を射影演算子によって記述できるという仮説であるが、測定演算子による測定から導ける。}
    はPOVM測定であり、物理的に実現可能である。とくに、PVM測定は射影測定と等価である。
  \end{itemize}
\end{frame}

\begin{frame}
  \frametitle{最も一般的な測定:CP-インストルメント(測定演算子)による測定}
  POVM測定は、物理量のとある測定値の確率分布を得るものであり、測定後の状態は得ることができない。
  そのため、測定過程まで含めた測定が最も一般的な測定である。状態$\rho$による測定$M$によって値$m$が得られた後の状態を$\rho_m$とする。
  このとき、アフィン性を満たし、かつ完全正写像(CP-map)である写像$\Lambda_m\colon \mathcal{S}(\mathcal{H})\mapsto \mathcal{L}(\mathcal{H})$を
  \begin{equation}
    \Lambda_m\colon \rho \mapsto \rho_m' = \Pr(M=m\mid \rho)\rho_m
  \end{equation}
  と定める。ざっくりいうと、測定前の状態から測定後の状態を与えるような写像である。(規格化しないことによってアフィン性を満たす。)
  また、$\Lambda(\rho)\coloneqq \sum_m \Lambda_{m}(\rho)$とすると、
  \begin{equation}
    \Tr(\Lambda(\rho)) = \sum_{m}\Tr(\rho) = \sum_{m}\Pr(M=m\mid \rho) = 1 = \Tr(\rho)
  \end{equation}
  となるため、トレース保存のCP-mapである。このようなCP-mapをTPCP(CPTP)-mapという。
  TPCP-mapを構成する$\Lambda_m$をCP-インストルメントという。また、$\Lambda$がCP-mapである同値条件として、
  \begin{equation}
    \Lambda(A) = \sum_{k=1}^{l}V_kAV_k^{\dagger},\qq{$V_k$は線形演算子}
  \end{equation}
  がある。とくに、CPTP-mapは$\sum_k V_kV_k^{\dagger}=I$という条件を付ければ上のような表現ができ、
  特にKraus表現と呼ぶ。
\end{frame}

\begin{frame}
  \frametitle{CP-インストルメントと測定演算子}
  CP-インストルメント$\Lambda_m$は、$l_m\leq d_Ad_B$個の測定演算子$V_{k}^{(m)}\colon \mathcal{H}_A\mapsto \mathcal{H}_B$を用いて
  \begin{equation}
    \Lambda_m(\rho) = \sum_{k=1}^{l_m}V_{k}^{(m)}\rho V_{k}^{(m)\dagger}
  \end{equation}
  と表現できる。したがって、測定演算子の組を持ってきて、測定の確率分布
  \begin{equation}
    \Pr(M=m\mid \rho) = \Tr(\Lambda_m(\rho)) = \sum_{k=1}^{l_m}\Tr(V_{k}^{(m)}\rho V_{k}^{(m)\dagger})
  \end{equation}
  を与えることができる。また、測定後の状態は
  \begin{equation}
    \rho_m = \frac{\Lambda_m(\rho)}{\Pr(M=m\mid \rho)} = \frac{\sum_{k=1}^{l_m}V_{k}^{(m)}\rho V_{k}^{(m)\dagger}}{\sum_{k=1}^{l_m}\Tr(V_{k}^{(m)}\rho V_{k}^{(m)\dagger})}
  \end{equation}
  と求まる。このように、測定演算子を用いると、測定後の状態を求めることができる。さらに、測定演算子による測定の結果はCP-インストルメントによる測定と等価であることも示される。
  そして、任意のCP-インストルメントは間接測定によって物理的に実現できることも示される。
\end{frame}

\begin{frame}
  \frametitle{測定演算子の例}
  \begin{itemize}
    \item POVM測定は、測定演算子による測定から導ける。$V_m = \sqrt{E_m}$とすると、$\{V_m\}_{m\in \mathcal{M}}$は測定演算子の組となっている。
    このとき、
    \begin{align}
      \Pr(M=m\mid \rho) &= \Tr(V_m\rho V_m^{\dagger}) = \Tr(E_m\rho) \\
      \rho_m & = \frac{\sqrt{E_m}\rho \sqrt{E_m}}{\Tr(E_m\rho)}
    \end{align}
    となる。ユニタリ演算子$U$によって$V_m = U\sqrt{E_m}$としても、測定の確率分布は変わらない。しかし、$\rho_m$は変わってしまう。
    つまり、POVMは測定後の状態を一意に決定しないということがわかる。
    \item $V_m = P_m = P_m^{\dagger} = P_m^2$とすると、$\{V_m\}_{m\in \mathcal{M}}$は測定演算子の組となっている。
    このとき、
    \begin{align}
      \Pr(M=m\mid \rho) &= \Tr(V_m\rho V_m^{\dagger}) = \Tr(P_m\rho P_m) = \Tr(P_m\rho) \\
      \rho_m & = \frac{P_m\rho P_m}{\Tr(P_m\rho)}
    \end{align}
    となる。これは、Von Neumannの射影仮説の一般化となっている。$\rho=\ketbra{\psi}$とすると、それを再現できる。
  \end{itemize}

 
  

\end{frame}



\end{document}