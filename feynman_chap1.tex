\RequirePackage{plautopatch}
\documentclass[11pt,aspectratio=169,xcolor=dvipsnames,table,dvipdfmx]{beamer}
\usepackage{mybeamer}
%タイトル
\title{Feynman-Hibbs「経路積分と量子力学」}
\subtitle{第一章 量子力学の基本概念}
\author{\textbf{松本侑真}}
\date{\today}

\begin{document}
\maketitle

\begin{frame}{量子力学における確率}
  20世紀の初め頃から、実験物理学の分やで古典力学の妥当性を否定するような奇妙な現象が続々と蓄積されていた。
  これらの新しい現象を説明できるような新しい理論の枠組みでは、電子が波動・粒子の両方の性質を持つように見える。
  このような事態は1926年、1927年に{\color{red}量子力学}と呼ばれる理論によって完全に解決された。
  \begin{block}{量子力学による予言}
      量子力学によると、
    \begin{itemize}
      \item 精密な結果を予測することが根本的に不可能な実験が存在するため、
      いろいろな確率の確率を計算することで満足しなければならない。
      \item 自然の法則の中にある確率は、Laplaceの古典確率論における確率組み合わせの放送と異なるものである。
      \item 量子力学の法則は、実験の対象の大きさが増大すればするほど、Laplaceの法則に近くなる。
    \end{itemize}
      ということがわかってきた。
  \end{block}
\end{frame}

\begin{frame}{1つの概念上の実験}
  二重スリット実験は、量子力学の効果が現れる最も単純な実験の1つである。
  電子が粒子のように振舞うということから、
  \begin{itemize}
    \item 線源$S$からスクリーン上のとある位置$x$に進む電子はそれぞれ穴1、穴2どちらか一方を通る。
    \item その帰結として、$x$に到達する確率は、穴1を通って到達する確率$P_1$と、穴2を通って到達する確率$P_2$の和に等しい
    と予想される。
  \end{itemize}
  しかし、実際の実験では、両方の穴が開いているときに電子が$x$に到達する確率$P(x)$に対して、
  \begin{equation}
    P(x) \neq P_1 + P_2
  \end{equation}
  となっている。そのため、先ほどの予想は誤りであることになる。
\end{frame}

\begin{frame}
  \frametitle{確率振幅}
  \begin{itemize}
    \item   両方の穴が開いているときに$x$に到達する確率は、片方の穴のみが開いているときの確率の和ではない。
    実際に、$P(x)$は線源$S$を出発した波が2つの穴を通ってスクリーンに到達したときに生じる干渉パターンの強度分布として知られている。
    \item 波動を表現するのに最も簡単な方法は複素数を用いることであり、$P(x)$は、$x$に到達することを表すある複素数$\phi(x)$の絶対値の2乗として表現される。
  \end{itemize}
  \begin{block}{2重スリット実験に関する法則}
    複素数$\phi_1,\,\phi_2$に対して、
    \begin{equation}
      P = \abs{\phi}^2 = \abs{\phi_1 + \phi_2}^2 ,\quad P_1 = \abs{\phi_1}^2 ,\quad P_2 = \abs{\phi_2}^2
    \end{equation}
    が成り立つ。
  \end{block}
  $\phi_1$は、ある波動方程式に対して、線源$S$から穴1に伝播し、そして穴1から$x$へと伝播する波を表現する解として求められる。
  つまり、$\phi_1$は電子(光の場合は光子)の波動性を表している。\\
  \textbf{まとめ: }\color{red}われわれは$x$に在る検出装置に到達する波の強度を計算し、その強度を粒子の到達確率として解釈する。
\end{frame}

\begin{frame}
  \frametitle{2重スリット実験の考察}
  \begin{block}{論理的困難}
    波動性と粒子性を両方用いることは何ら矛盾を生じない。ただし、実験の状況についてどのような言明が許されるかには細心の注意が必要である。
\begin{itemize}
  \item 2重スリット実験に関する法則によると、一般に$P=P_1+P_2$は成り立たない。
  \item このことから、2つの穴が開いているとき、粒子がどちらか一方の穴を通ることは正しくないという結論にせざるを得ない。
  \item 仮に、粒子がどちらか一方の穴を通るということが正しいとすると、穴1と穴2を通るのは排反事象であることから、$P=P_1+P_2$が成立してしまう。
\end{itemize}
  \end{block}
古典的描像に基づく論理的困難を乗り越えるために、様々な実験の工夫を行うことができる。
例えば、光子を用いる場合では、干渉性を持つ2つの経路を何センチも大きく話すことによって、2つの光の経路をほとんど確かに独立にすることができる。
しかし、実際の状況が想像以上に深遠であることは、次の実験によって明らかとなる。
\end{frame}

\begin{frame}
  \frametitle{電子の経路の観測}
  \begin{itemize}
    \item 電子がどちらか一方の穴を通るのではないということを検証することは、実験の工夫によって可能である。
    \item 電子は光を散乱するため、穴の後方に光源を置いて光の散乱位置を観測すれば良い。
  \end{itemize}
  \begin{block}{観測の効果}
    電子がどちらの穴を通るかを観測し、同時に電子が$x$に到達する確率を測定してみる。
    こうすれば、$x$に到達する電子が穴1、穴2のどちらを通ってきたのかを実験的に確かめることが可能になる。その結果、
    \begin{itemize}
      \item 穴1を通る電子は、$P_1$で与えられることが確かめられる。
      なぜなら、$x$に到達する電子のうちで、穴1を通ったと思われる(光の散乱から確かめられる)電子のみを選べば、
      確かに$P_1$の分布になっていることがわかるからである。この結果は、穴2が閉じているかどうかによらず得られるものであり、穴2の存在は穴1の近くの運動に影響を与えない。
      \item 穴2を通る電子についても同様である。
      \item それらの電子を一緒にすれば、$P=P_1+P_2$が得られるはずであり、実際に実験すればそうなる。
      この場合、$P = \abs{\phi_1+\phi_2}^2$とはならない。
    \end{itemize}
  \end{block}
\end{frame}

\begin{frame}
  \frametitle{観測の結果の影響について}
  \begin{block}{光による撹乱}
    \begin{itemize}
      \item 光はエネルギー$h\nu$、運動量$h/\lambda$の光子として入射してくる。光を弱くするということは光子の数を少なくするということであり、散乱して観測できる電子の個数が減るというだけである。
      \item 電子を見出した時には、完全に1個の光子が散乱されたことになり、$h/\lambda$程度の有限な運動量が電子に与えられることになる。
      \item 観測された電子は$P_1+P_2$、観測されなかった電子は$\abs{\phi_1+\phi_2}^2$としてスクリーン上に分布する。
    \end{itemize}
    この場合、全体の分布はそれぞれの分布の加重平均($p$は観測される電子の個数に比例)
    \begin{equation}
      P = (1-p)\abs{\phi_1+\phi_2}^2 + p(P_1+P_2)
    \end{equation}
    となる。また、$\lambda$を長くすれば撹乱は小さくなるが、波長$\lambda$の光源の位置を$\lambda$の程度よりも精密に確定することができない。
    したがって、電子が通った穴の位置を決定することもできない。
  \end{block}
\end{frame}

\begin{frame}
  \frametitle{不確定性原理}
  Heisenbergは、「新しい力学が無矛盾であるためには実験を行う際の撹乱を小さくするのに限界がある」と述べ、観測による撹乱の影響に初めて気づいた。
  \begin{block}{Heisenbergによる不確定性原理の表現}
    \begin{itemize}
      \item 古典物理学では、粒子はある特定の軌道に沿って運動するものと記述され、任意の時刻においてたとえば位置と運動量が正確に与えられる。
      \item しかし、量子力学に従う系(現実の系)では、古典物理学的見地から見ると、位置もしくは運動量が特定できないで不確定な系として見える。
      \item 別の測定手段によって位置や運動量片方の不確定性を減らすことはできるが、両者を同時
      \footnote{ここでいう「同時」とは、「同時刻」ではなく、「2つの物理量の同時確率分布に対してそれぞれの周辺確率分布が正しく定まるかどうか」ということだと思います。}
      に正確に知ることはできないのである。
      \item どのような実験においても、運動量と位置の不確定性の積は$\hbar$の程度よりも小さくなることはない。
    \end{itemize}
  \end{block}
\end{frame}
\begin{frame}{不確定性原理が述べていること}
    \begin{itemize}
      \item スクリーンCの穴1と穴2を通って$x$に到達する電子では、軌道が曲がることによる運動量の変化分が異なる。
      それぞれの穴を通るときの運動量の変化分の差を$\delta p$とする。
      \item $C$が上下方向に自由に動けるものとして、その運動量変化の大きさを$\delta p$以上の精度で測定すれば、電子がどちらの穴を通ったかがわかる。
      \item 実際にその精度で測定できる装置を用いれば、干渉は$P=P_1+P_2$となる。では、なぜ$P=\abs{\phi_1+\phi_2}^2$とならないのか?
      \begin{itemize}
      \item $C$が上下に$\delta x$だけ動くと、電子の干渉面への到達位置は上下に$\delta x$だけ変化する。
      \item その結果、干渉面の位置$x$の上下$\delta x$の間の平均値として干渉パターンが観測されるため、到達位置の較正をしないと$C$を固定した場合と同じ分布が得られない。
      \item すなわち、スクリーン$C$の位置の変化$\delta x$は、干渉縞の極大点の幅$d$の半分以上の精度$(\delta x <d/2)$で決定してスクリーン$C$のゼロ点を較正する必要がある。
      \item 不確定性原理が正しいとすると、スクリーンの運動量変化を$\delta p$以上の精度で測定できる場合は、$\delta x$を$d/2$以上の精度で測定できないことを意味している。
      \end{itemize}
    \end{itemize}
  以上で不確定性原理が証明されたわけではないが、実験的に不確定性原理が破れていることが示されておらず、
  また、現実世界を高い精度で予言している量子力学の法則が首尾一貫したものであるためには、不確定性が要求されるように思われている。
\end{frame}

\begin{frame}
  \frametitle{スクリーンの位置による干渉縞の消失}
\begin{center}
  \adjincludegraphics[height=0.5\linewidth]{pic/hukakutei.jpg}
\end{center}
\end{frame}

\begin{frame}
  \frametitle{確率概念のまとめ}
  線源$S$から検出器の位置$x$へ進む電子は、スクリーンC上の穴1を通って$x$に到達する場合の振幅と、
  穴2を通って到達する場合との2つの振幅を持つ。さらに、それぞれの選択肢の振幅を加えることにより、全部の事象に対応させることができる。
  たとえば、$x$に到達することの全振幅は
  \begin{equation}
    \phi(x) = \phi_1(x) + \phi_2(x)
  \end{equation}
  であり、その事象の起こる確率は
  \begin{equation}
    P(x) = \abs{\phi(x)}^2 = \abs{\phi_1(x)+\phi_2(x)}^2
  \end{equation}
  である。その事象が完了する前に、粒子の状態を観測してしまうと、その全振幅の構成を撹乱することになる。
  すなわち、粒子の系が他の状態にある可能性を排除したことになり、排除された状態に付随した確率振幅は全振幅の計算には加えられない。
\end{frame}
\begin{frame}
  \frametitle{スクリーンにさらに穴を開け、スクリーンの数も増やす}
  \begin{itemize}
    \item 先ほどの考え方を拡張して、地点AB間に置くスクリーンの数を増やし、スクリーン自体の穴もさらに開けると、それぞれの穴を通る振幅の和を考える必要がある。
    \item 例えば、位置$y_E$にあるスクリーン$E$の高さ$x_E$の穴から出て、位置$y_D$にあるスクリーン$D$の高さ$x_D$の穴に入る電子の振幅を考える必要がある。
    \item すなわち、$y_E$から$y_D$への経路の選択肢の和は、位置$y_D,\,y_E$にある2つのスクリーンを通過する高さを記述する連続パラメータ$x_D,\,x_E$についての積分によって与えられる。
    \item スクリーンの数も無限個に増やすことで、地点ABを結ぶ経路の和は、$y_A\leq y\leq y_B$と$x_A\leq x\leq x_B$による二重積分で表現される。
  \end{itemize}
  また、経路$x(y)$上の区間の各点を通る時刻も考えることができる。
  すなわち2次元空間であれば、2つの関数$x(t),\,y(t)$が与えられれば1つの経路が決定される。
  このようにいて、特定の経路$x(t),\,y(t)$をとる振幅という概念が得られる。
\end{frame}

\frame{\centering{\Huge{Thank you for your attention!}}}




\end{document}